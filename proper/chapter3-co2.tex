%======================================================================
\chapter{Photocatalysis of \texorpdfstring{CO\textsubscript{2}}{CO2}}
\markright{Photocatalysis of \texorpdfstring{CO\textsubscript{2}}{CO2}}
%======================================================================

%======================================================================
\section{Introduction}
%======================================================================

Only 6 years after \ce{Re^I} complexes using 2,2'-bipyridine were characterized, Hawecker, Ziessel\todo{check ref names}, and Lehn showed the effectiveness of the compound for the photocatalytic reduction of \ce{CO2}\autocite{hawecker1983}. Since then, many have shown the efficacy of a wide range of $\alpha$-diimino complexes for the reaction\todo{a few references}, and expansion of the systems to bimetallic complexes with ruthenium and osmium as electron transfer agents has produced a wide range of results\todo{refer to these as well. Red Light paper?}. The mechanism of reduction has been subject of some debate: while mechanisms have been proposed since Lehn et. al. soon after their original publication\autocite{hawecker1986}, modifications have been submitted routinely over the past decades\todo{references, Morris, JACS dimer, mech. basis}. Further discussion and a proposal of a new mechanism geometry based on computational and experimental data can be read in \autoref{chap.mech}.

%======================================================================
\section{Photocatalytic Reactions with New Compounds}
%======================================================================

The photocatalytic cycle is, simply, a photon-induced \gls{ac.mlct}, followed by the extraction of an electron from a sacrificial reductant. This radical, negatively charged species sheds the anion, opening up a reaction site. Reaction between a \ce{CO2}, a proton (from the decomposition of the reductant or elsewhere), and the catalyst yields any number of \ce{CO}, \ce{H2O}, formate (\ce{HCO2-}), or carbonate (\ce{CO3H-}), depending on the mechanistic pathway. 

%----------------------------------------------------------------------
\subsection{Conditions}
%----------------------------------------------------------------------

Reaction conditions in use in literature have remained typically unchanged since the original papers. A mixture of \gls{ac.dmf} with either \gls{ac.teoa} or \gls{ac.tea} at a 5:1 ratio is used to make a 1.0 mM solution of catalyst, with `excess' (depending on reference, a 1.1 to 25 molar ratio) electrolyte salt (typically \ce{Et4NX} or \textit{t}-\ce{Bu4NX}, where X = halide from catalyst) added as a stabilizer. Solutions are degassed by bubbling of \ce{CO2} and a consistent headspace is left to form over the solution. The reaction is monitored via \gls{ac.gc} analysis of the headspace, using a HP gas chromatograph with a 15 m CARBONPLOT column with 0.320 mm inner diameter and 1.50 $\mu$m film in a 40$^\circ$C oven. The instrument is fitted with a \gls{ac.tcd}, and, while using He as a carrier gas, is able to resolve \ce{CO} and \ce{CO2} completely.  

%----------------------------------------------------------------------
\subsection{Experimental Results}
%----------------------------------------------------------------------

Both bidentate and terdentate \ce{$\kappa$^n(terpy)Re(CO)_{5-n}X} (n=2, 3) \textbf{2.1} and \textbf{2.2} complexes show no activity for \ce{CO2} reduction. Modification of testing time, light source, product analysis methods, solvent, sacrificial reductant, pH, presence of electrolyte, presence of \ce{H2O}, or variation of anion (X=Cl, Br, OTf, CN) shows no change of this inactivity. Testing of \ce{$\kappa$^2(bipy)Re(CO)3Cl} under the same reaction and testing conditions shows production of approximately 6 mL \ce{CO} from \ce{CO2} (~20\% conversion) in 1 hour of photolysis with visible ($\lambda$ \textgreater 400 nm) light, verifying the method, isolating the catalyst as the ineffective species. 

%======================================================================
\subsection{Rationalization of Results}
%======================================================================

Examples in literature \todo{reference} show the requirement for fluorescence for successful catalytic candidates. The explanation for this is the requirement for a stable, long-lasting excited state, with lifetime greater than that of the timescale required for electron abstraction from the sacrificial amine. The observed fluorescence demonstrates the lack of non-radiative relaxation pathways, considered to be an analogue for the extended lifetime of the excited state. Sample \textbf{2.2} shows only poor fluorescence. While the complex is able to absorb light across the spectrum, and has \gls{ac.homo} to \gls{ac.lumo} transitions with high enough energy\footnote{Electrochemical reduction in similar environments takes 1.7-2.1 V, equivalent to \gls{ac.homo}-\gls{ac.lumo} transitions from 590-750 nm\autocite{grills2014}.} for the catalyzed reduction of \ce{CO2} to \ce{CO}, it appears as if the catalysis is not initiated due to a short excited state lifetime. Fluorescence data presented in \nameref{chap.newchem}, at \autoref{ss.fluorescence} shows the lack of fluorescence in terdentate samples \textbf{2.2} and \textbf{2.4}.

Explanation for the lack of \ce{CO} production observed in the attempted photochemical reduction of \ce{CO2} by bidentate sample \textbf{2.1} and others must come from another angle. These samples are quite fluorescent, emission from the powder sample can be seen with the naked eye with simulation by a 405 nm laser pen under ambient light environments. Importantly, two clues come from the reaction mixture: under intense visible light in the presence of \ce{CO2} the compound bleaches and colour does not return after storage in the dark, bubbling of new \ce{CO2}, or other manipulations. Secondly, a mixture of sacrificial amine, \gls{ac.dmf}, \ce{Et4NCl}, and catalyst \textbf{2.1} in ratios identical to what is required in the reaction mixture changes colour from a pale yellow to a deep red irreversably after 5 days at ambient temperature, in dark and under ambient light. \todo{TEOA insertion, maybe clear is a acid, formate, carbonate, tricarbonylcation or other complex?}



