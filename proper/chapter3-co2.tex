%======================================================================
\chapter{Photocatalysis of \texorpdfstring{CO\textsubscript{2}}{CO2}}
\markright{Photocatalysis of \texorpdfstring{CO\textsubscript{2}}{CO2}}
%======================================================================

%======================================================================
\section{Introduction}
%======================================================================

Only 6 years after \ce{Re^I} complexes using 2,2'-bipyridine were characterized, Hawecker, Ziessel\todo{check ref names}, and Lehn showed the effectiveness of the compound for the catalytic photoreduction of \ce{CO2}\autocite{hawecker1983}. Since then, many have shown the efficacy of a wide range of $\alpha$-diimino complexes for the reaction\todo{a few references}, and expansion of the systems to dimetallic complexes with ruthenium and osmium as electron transfer agents has produced a wide range of results\todo{refer to these as well. Red Light paper?}. The mechanism of reduction has been subject of some debate, while mechanisms have been proposed since Lehn et. al. soon after their original publication\autocite{hawecker1986}, modifications have been submitted routinely over the past decades\todo{references, Morris, JACS dimer, mech. basis}. Further discussion and a proposal of a new mechanism geometry based on computational and experimental data can be read in \autoref{chap.mech}.

%======================================================================
\section{Photocatalytic Reactions with New Compounds}
%======================================================================

The photocatalytic cycle is, simply, a photon-induced \gls{ac.mlct}, followed by the extraction of an electron from a sacrificial reductant. This radical, negatively charged species sheds the anion, opening up a reaction site. Reaction between a \ce{CO2}, a proton (from the decomposition of the reductant or elsewhere), and the catalyst yields any number of \ce{CO}, \ce{H2O}, formate (\ce{HCO2-}), or carbonate (\ce{CO3H-}), depending on the mechanistic pathway. 

%----------------------------------------------------------------------
\subsection{Conditions}
%----------------------------------------------------------------------

Reaction conditions in use in literature have remained typically unchanged since the original papers. A mixture of \gls{ac.dmf} with either \gls{ac.teoa} or \gls{ac.tea} at a 5:1 ratio is used to make a 1.0 mM solution of catalyst, with 'excess' (a 1.1 to 25 molar ratio) electrolyte salt (typically \ce{Et4NX} or \textit{t}-\ce{Bu4NX}, where X = halide in catalyst) added as a stabilizer. Solutions are degassed by bubbling of \ce{CO2} and a consistent headspace is left to form over the solution. The reaction is monitored via \gls{ac.gc} analysis of the headspace, using a HP gas chromatograph with a 15 m CARBONPLOT column with 0.320 mm inner diameter and 1.50 $\mu$m film in a 40$^\circ$C oven. The instrument is fitted with a \gls{ac.tcd}, and, while using He as a carrier gas, is able to resolve \ce{CO} and \ce{CO2} completely.  

%----------------------------------------------------------------------
\subsection{Experimental Results}
%----------------------------------------------------------------------

Both bidentate and terdentate \ce{$\kappa$^n(terpy)Re(CO)_{5-n}X} (n=2,3) complexes show no activity for \ce{CO2} reduction. Modification of testing time, light source, product analysis methods, solvent, sacrificial reductant, pH, presence of electrolyte, presence of \ce{H2O}, or variation of anion (X=Cl, Br, OTf, CN) shows no change of this inactivity. Testing of \ce{$\kappa$^2(bipy)Re(CO)3Cl} under the same reaction conditions shows production of approximately 6 mL \ce{CO} from \ce{CO2} (20\% conversion) in 1 hour of photolysis with visible ($\lambda$ \textgreater 400 nm) light.

%While some literature has suggested that \ce{$\kappa$^2(terpy)Re(CO)3Cl} is an active photocatalyst when solvated in \gls{ac.dmso}\todo{ref}, this was not seen in our experiments. Various studies have shown the impact of the presence of electrolyte, \ce{H2O}, or 

% - - - - - - - - - - - - - - - - - - - - - - - - - - - - - - - - - - - 
\subsubsection{SubSubSection}
% - - - - - - - - - - - - - - - - - - - - - - - - - - - - - - - - - - -

% - - - - - - - - - - - - - - - - - - - - - - - - - - - - - - - - - - -
\subsubsection{SubSubSection}
% - - - - - - - - - - - - - - - - - - - - - - - - - - - - - - - - - - -

% - - - - - - - - - - - - - - - - - - - - - - - - - - - - - - - - - - -
\subsubsection{SubSubSection}
% - - - - - - - - - - - - - - - - - - - - - - - - - - - - - - - - - - -

%======================================================================
\section{Section}
%======================================================================

%======================================================================
\section{Section}
%======================================================================

%----------------------------------------------------------------------
\subsection{Subsection}
%----------------------------------------------------------------------
