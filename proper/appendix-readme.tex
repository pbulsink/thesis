\chapter{TurboControl and TurboGo Manual}\label{app.readme}
\markright{TurboControl User's Manual}


TurboControl is a series of scripts to run TurboMole jobs from Gaussian style inputs. The following is the user manual included with distributions of TurboControl 

\section{Introduction}
Gaussian software is well known for the user friendly GUI it contains (via GaussView). TurboMole, another computational suite, is known for its speed and optimizations, but has a significantly higher learning curve and is less beginner friendly. This software is an attempt to be able to use the user friendly input from Gaussian to smooth over the use of TurboMole.

\section{System Requirements}

There are two user-facing scripts available, both written to work with TurboMole 6.1-6.5 on clusters using Grid Engine queuing software. The only tests of operation are on a system with the following details:

\begin{itemize}
\itemsep1pt\parskip0pt\parsep0pt
\item
  Rocks 6.1 (Emerald Boa)/CentOS 6.3
\item
  Open Grid Scheduler/Grid Engine 2011.11p1
\item
  Python 2.7.3
\end{itemize}

Other systems, including different operating systems, different versions of Grid Engine or python, or on other systems, are not supported.

Python dependencies include:

\begin{itemize}
\itemsep1pt\parskip0pt\parsep0pt
\item
  pexpect 3.2\autocite{pexpect}
\item
  openbabel (optional)\autocite{openbabel, oboyle2011} 
\end{itemize}

Prior to running TurboGo or TurboControl, a valid installation of TurboMole must be available. On systems where computational modules must be loaded, TurboMole must have been loaded to the environment. Additionally, running the TurboMole environment configuration is recommended but not required prior to launching TurboGo or TurboControl:

\begin{center}
\begin{verbatim}
$ source $TURBODIR/Config_turbo_env
\end{verbatim}
\end{center}

\section{Installation}
Installation of TurboControl is very simple. Just extract the .tar.gz file available from the code repository at \url{https://github.com/pbulsink/turbocontrol/releases/latest}. Alternately, the source may be downloaded from \url{http://github.com/pbulsink/turbocontrol} and used without installation.

\section{TurboGo}

TurboGo is a script fun on an input file. It generates the inputs required for TurboMole jobs, and submits the job to the GridEngine queue before quitting. TurboGo is run with the following syntax:

\begin{center}
\begin{verbatim}
$ turbogo [-h] [-v] [-q] file
\end{verbatim}
\end{center}

positional arguments:

\begin{Verbatim}[baselinestretch=0.75]
file                  Read input from gaussian-type input FILE.
\end{Verbatim}

More info on the input files is available below.

optional arguments:

\begin{itemize}
\item \texttt{-h, --help            }Show this help message and exit
\item \texttt{-v, --verbose         }Run more verbose (show debugging info)
\item \texttt{-q, --quiet           }Run less verbose (show only warnings)
\end{itemize}

TurboGo saves a log file (turbogo.log) in the directory in which it is run. A second log file (define.log) will remain if the setup crashes or is terminated at some points, or if the script is run verbose.

TurboGo writes the final coordinates to final\_geometry.xyz. If openbabel is installed, it will also write finalgeom.mol. The entire optimization is written to optimization.xyz for viewing with a molecular viewer, such as vmd.

\section{TurboControl}

TurboControl is a management script called from a parent directory containing sub directories of input files. Each input file must be in its own directory. The input file format must be the same as the input format for TurboGo (listed above), with the extension `.in', `.inp', `.input', `.com', or `.gjf'. TurboControl reads the inputs and submits the jobs to the computational cluster queue. It then monitors running jobs to determine when the script has finished. If the job is an Opt-Freq, it prepares the frequency analysis and resubmits to the queue. TurboControl analyses completed Opt-Freq jobs for true optimization, and attempts to re-run jobs with modified geometries when Transition States are found. TurboControl will not get stuck on the same transition state, but will return a `stuck' job. TurboControl is run with the following syntax:

\begin{center}
\begin{verbatim}
$ turbocontrol [-h] [-v/-q] [-s]
\end{verbatim}
\end{center}

Optional arguments:

\begin{itemize}
\item \texttt{-h, --help            }Show this help message and exit
\item \texttt{-v, --verbose         }Run more verbose (show debugging info)
\item \texttt{-q, --quiet           }Run less verbose (show only warnings)
\item \texttt{-s, --solvent         }List available solvents for COSMO and quit
\end{itemize}

TurboControl outputs information every 3 hours on the status of the jobs. It writes a log file (\texttt{turbocontrol.log}) and may or may not leave other log files in each directory (depending on verbosity level). Ends when the last job finishes or crashes. Requires 1 node or can be run on the headnode (minimal resource consumption especially after initial job preparation and submission.)

TurboControl assists with analysis by outputting a \texttt{stats.txt} file as jobs complete. This file contains file details, optimization and frequency timing details, energy, and the first frequency. Additional information can be requested by including the \texttt{freeh} keyword (see below).

\section{Input File Format}

The input file format is similar to that well known by Gaussian users. A series of keywords, one per line and indicated by a `\%', is followed by the `route card' (specific job information). Charge and spin is indicated, then the molecule is shown in Cartesian format. This is followed by optional modifications to the TurboMole Control file. Note the location of blank lines in the example (Section 5.7).

\subsection{Keywords}

Keywords are as follows:

\begin{itemize}
\item \texttt{\%nproc} - number of processors to use for the calculation job.
  \begin{itemize}
    \item Synonym: \texttt{\%nprocessors}
  \end{itemize}
\item \texttt{\%arch} - parallelization architecture to use for the job.
  \begin{itemize}
    \item Synonyms: \texttt{\%architecture}, \texttt{\%para\_arch}
   \end {itemize}
\item \texttt{\%maxcycles} - number of optimization iterations before failing.
\item \texttt{\%autocontrolmod} - DEFAULT - modify the \texttt{control} file to include optimizations to speed up the job.
\item \texttt{\%nocontrolmod} - do not modify \texttt{control} file as above.
\item \texttt{\%rt} - specify max expected runtime (for any part of job)in hours. Allows backfilling in GridEngine queue to speed up job submission. For example, for a 1 hour opt and 4 hour freq, submit at least a \texttt{rt} of 4
\item \texttt{\%cosmo} - use TurboMole's COSMO solvation model with the specified solvent or \texttt{None} to use the idealized solvent (epsilon = infinity). List of available solvents can be shown by running \texttt{turbocontrol -s}
\end{itemize}

Gaussian args, including \texttt{\%nosave}, \texttt{\%rwf={[}file{]}}, \texttt{\%chk={[}file{]}}, and \texttt{\%mem={[}memory{]}} are silently ignored.

\subsection{Route Card Options}

Route cards take the form of the following:

\texttt{\# {[}jobtype(s){]} {[}joboption(s){]}}

Job types available:

\begin{itemize}
\item \texttt{opt} - Perform a geometry optimization
\item \texttt{freq} - Perform a frequency analysis. Specify method via numforce or aoforce. default = numforce
\item \texttt{sp} - Perform a single point energy calculation.
  \begin{itemize}
    \item Cannot be combined with Opt or Freq
  \end{itemize}
\item \texttt{ts} - Perform a transition state search to find 1 imaginary vibration. 
  \begin{itemize}
    \item Cannot be combined with Opt or Freq
  \end{itemize}
\item \texttt{prep} - Prepare the job but do not submit to queue.
  \begin{itemize}
    \item Cannot be combined with Opt or Freq
  \end{itemize}
\end{itemize}

Job options available:

\begin{itemize}
\item \texttt{ri} - Use TurboMole's ri approximation
\item \texttt{marij} - Use TurboMole's marij approximation 
  \begin{itemize}
    \item Requires \texttt{ri}
  \end{itemize}
\item \texttt{disp} - Use TurboMole's implementation of Grimme's dispersion, version 3
\item \texttt{aoforce} - Use aoforce for frequency jobs
\item \texttt{numforce} - Use numforce for frequency jobs
\item \texttt{freeh} - Use TurboMole's \texttt{freeh} thermodynamics data script to extract thermodynamic information after frequency analysis
\end{itemize}

\subsection{Title}

Following the Route cards, a blank line is added, then a line containing the title of the calculation. This can include any characters, spaces, etc., remaining on only one line. This is followed by a blank line.

\subsection{Charge and Spin}

Charge and spin are listed as two numbers separated by a space: charge spin (eg:0 1)

\subsection{Geometry}

Geometry in xyz coordinate format: Element xcoord ycoord zcoord. Z-matrix geometry is not supported by TurboControl or TurboGo.

\subsection{Additional control File Modifications}

Additional lines to be added or removed from control. Lines automatically added are, as required,:

\begin{Verbatim}[baselinestretch=0.75]
$ricore 0
$paroptions ga_memperproc 900000000000000 900000000000
$parallel_parameters  maxtask=10000
$ricore_slave 1
$maxcor 2048
\end{Verbatim}

Additional lines may be added, or lines removed, by placing them after the geometry with a \texttt{\$} (for addition) or \texttt{-\$} (for removal).

\subsection{Example Input File}

An example input file for benzene in dmf:

\begin{Verbatim}[baselinestretch=0.75]
%nproc=4
%arch=GA
%maxcycles=250
%rt=6
%cosmo=dmf
# opt freq b3-lyp/def2-TZVP ri marij numforce

Benzene Optimization & Frequency

0 1
C  0.000  1.396  0.000
C  1.209  0.698  0.000
C  1.209 -0.698  0.000
C  0.000 -1.396  0.000
C -1.209 -0.698  0.000
C -1.209  0.698  0.000
H  0.000  2.479  0.000
H  2.147  1.240  0.000
H  2.147 -1.240  0.000
H  0.000 -2.479  0.000
H -2.147 -1.240  0.000
H -2.147  1.240  0.000

$disp
-$paraoptions

\end{Verbatim}


\section{Code Details}\label{sec.code}

Coverage percentages of code unit tests are listed in \autoref{tab.codetest}. Results are low for def\_op, screwer\_op, cosmo\_op, freeh\_op, turbocontrol, and turbogo because they contain many lines of interacting with GridEngine or TurboMole. Testing is performed via monitoring the status of the scripts as they run in real conditions.

The code style is graded by PyLint and results are shown in \autoref{tab.pylint}. PyLint describes coding style, adherence to guidelines, and readability. It does not describe code efficiency or usefulness.

\begin{table}[!htb]
  \centering
    \caption{Test Coverage of scripts in TurboControl.}
    \begin{tabular}{crrrr}
    \toprule
    Name  & Statements & Missing & Excluded & Coverage \\
    \midrule
    cosmo\_op & 106   & 70    & 1     & 34\% \\
    cosmo\_op\_test & 17    & 1     & 0     & 94\% \\
    def\_op & 302   & 226   & 1     & 25\% \\
    def\_op\_test & 20    & 1     & 0     & 95\% \\
    freeh\_op & 162   & 55    & 1     & 66\% \\
    freeh\_op\_test & 27    & 1     & 0     & 96\% \\
    screwer\_op & 71    & 25    & 1     & 65\% \\
    screwer\_op\_test & 11    & 1     & 0     & 91\% \\
    test\_all & 18    & 0     & 0     & 100\% \\
    turbocontrol & 537   & 319   & 0     & 41\% \\
    turbocontrol\_test & 245   & 24    & 0     & 90\% \\
    turbogo & 343   & 132   & 0     & 62\% \\
    turbogo\_helpers & 383   & 52    & 0     & 86\% \\
    turbogo\_helpers\_test & 274   & 2     & 0     & 99\% \\
    turbogo\_test & 98    & 1     & 0     & 99\% \\ \midrule
    \textbf{TOTAL} & 2614  & 910   & 4     & 65\% \\
    \bottomrule
    \end{tabular}%
  \label{tab.codetest}
\end{table}%

\begin{table}[!htb]
  \centering
  \caption{PyLint Scores for Turbocontrol Code.}
    \begin{tabular}{cr}
    \toprule
    File  & Score /10 \\
    \midrule
    test\_all.py & 2.22 \\
    turbogo.py & 8.80 \\
    turbogo\_test.py & 6.97 \\
    turbocontrol.py & 8.55 \\
    turbocontrol\_test.py & 7.18 \\
    turbogo\_helpers.py & 8.81 \\
    turbogo\_helpers\_test.py & 7.45 \\
    def\_op.py & 8.18 \\
    def\_op\_test.py & 5.71 \\
    screwer\_op.py & 7.36 \\
    screwer\_op\_test.py & 6.67 \\
    freeh\_op.py & 8.71 \\
    freeh\_op\_test.py & 6.79 \\
    cosmo\_op.py & 8.22 \\
    cosmo\_op\_test.py & 6.67 \\
    \bottomrule
    \end{tabular}%
  \label{tab.pylint}
\end{table}%

\section{Citing TurboControl}

TurboControl, Turbogo, or any other parts of this code may be cited as:

Bulsink, Philip. TurboControl, v. 1.1.0. http://github.org/pbulsink/turbocontrol (accessed June 2014)

Change the version number to match the version that you used, and change the accessed date to when you installed or downloaded TurboControl. 

\section{License}

All third party software is a registered trademark of their respective creators. Use of third party software via this software is limited by the conditions as laid out by the respective companies. License to use this software in no way acts as a license to use any other separate referenced software.

The MIT License (MIT)

Copyright \copyright~2014 Philip Bulsink

Permission is hereby granted, free of charge, to any person obtaining a copy of this software and associated documentation files (the ``Software''), to deal in the Software without restriction, including without limitation the rights to use, copy, modify, merge, publish, distribute, sublicense, and/or sell copies of the Software, and to permit persons to whom the Software is furnished to do so, subject to the following conditions:

The above copyright notice and this permission notice shall be included in all copies or substantial portions of the Software.

THE SOFTWARE IS PROVIDED ``AS IS'', WITHOUT WARRANTY OF ANY KIND, EXPRESS OR IMPLIED, INCLUDING BUT NOT LIMITED TO THE WARRANTIES OF MERCHANTABILITY, FITNESS FOR A PARTICULAR PURPOSE AND NONINFRINGEMENT. IN NO EVENT SHALL THE AUTHORS OR COPYRIGHT HOLDERS BE LIABLE FOR ANY CLAIM, DAMAGES OR OTHER LIABILITY, WHETHER IN AN ACTION OF CONTRACT, TORT OR OTHERWISE, ARISING FROM, OUT OF OR IN CONNECTION WITH THE SOFTWARE OR THE USE OR OTHER DEALINGS IN THE SOFTWARE.

