% An appendix
%======================================================================
\chapter{Reaction Potential Energy Diagrams} \label{app.energy}
\markright{Reaction Potential Energy Diagrams}
%======================================================================

Potential energy diagrams for the reactions discussed in \autoref{chap.mech} are listed below in \Cref{fig.apes_all,fig.apes_axial,fig.apes_planar,fig.apes_formate,fig.apes_wgs,fig.apes_carbonate,fig.apes_gas,fig.apes_dmf,fig.apes_eximer}. 

\begin{landscape}

\begin{figure}[!htbp]
 \begin{center}
  \includegraphics[clip=true, height=142mm, keepaspectratio]{images/pes_all.eps}
 \end{center}
\caption[An overview of the energies of the three mechanistic pathways of photochemical \ce{CO2} reduction.]{An overview of the energies of the three mechanistic pathways of photochemical \ce{CO2} reduction.}
\label{fig.apes_all}
\end{figure}

\begin{figure}[!htbp]
 \begin{center}
  \includegraphics[clip=true, height=142mm, keepaspectratio]{images/pes_eximer.eps}
 \end{center}
\caption[Potential Energy Surface for the production of the excimer.]{Potential Energy Surface for the production of the excimer. Gas phase energies in red, DMF solvated energies in blue.}
\label{fig.apes_eximer}
\end{figure}

 
\begin{figure}[!htbp]
 \begin{center}
  \includegraphics[clip=true, height=142mm, keepaspectratio]{images/pes_axial.eps}
 \end{center}
\caption[Potential Energy Surface for the axial geometry of the water-gas shift mechanistic pathway.]{Potential Energy Surface for the axial geometry of the water-gas shift mechanistic pathway. Gas phase energies in red, DMF solvated energies in blue.}
\label{fig.apes_axial}
\end{figure} 

\begin{figure}[!htbp]
 \begin{center}
  \includegraphics[clip=true, height=142mm, keepaspectratio]{images/pes_planar.eps}
 \end{center}
\caption[Potential Energy Surface for the planar geometry of the water-gas shift mechanistic pathway.]{Potential Energy Surface for the planar geometry of the water-gas shift mechanistic pathway. Gas phase energies in red, DMF solvated energies in blue.}
\label{fig.apes_planar}
\end{figure} 

\begin{figure}[!htbp]
 \begin{center}
  \includegraphics[clip=true, height=136mm, keepaspectratio]{images/pes_wgs.eps}
 \end{center}
\caption[Potential Energy Surface for the two water-gas shift mechanistic pathway geometries.]{Potential Energy Surface for the two water-gas shift mechanistic pathway geometries. Axial geometry energies are shown in purple (gas phase) and green (solvated phase), and planar geometries shown in red (gas phase) and blue (solvated phase).}
\label{fig.apes_wgs}
\end{figure} 

\begin{figure}[!htbp]
 \begin{center}
  \includegraphics[clip=true, height=142mm, keepaspectratio]{images/pes_formate.eps}
 \end{center}
\caption[Potential Energy Surface for the formate mechanistic pathway.]{Potential Energy Surface for the formate mechanistic pathway. Gas phase energies in red, DMF solvated energies in blue.}
\label{fig.apes_formate}
\end{figure} 

\begin{figure}[!htbp]
 \begin{center}
  \includegraphics[clip=true, height=142mm, keepaspectratio]{images/pes_carbonate.eps}
 \end{center}
\caption[Potential Energy Surface for the bicarbonate mechanistic pathway.]{Potential Energy Surface for the bicarbonate mechanistic pathway. Gas phase energies in red, DMF solvated energies in blue.}
\label{fig.apes_carbonate}

\end{figure} 
\begin{figure}[!htbp]
 \begin{center}
  \includegraphics[clip=true, height=136mm, keepaspectratio]{images/pes_gas.eps}
 \end{center}
\caption[An overview of the energies of the three mechanistic pathways of photochemical \ce{CO2} reduction in gas phase.]{An overview of the energies of the three mechanistic pathways of photochemical \ce{CO2} reduction in gas phase. Excimer formation is shown in blue, the planar water-gas shift mechanism in red, the axial water-gas shift in orange, the bicarbonate in purple, and the formate in green.}
\label{fig.apes_gas}
\end{figure} 

\begin{figure}[!htbp]
 \begin{center}
  \includegraphics[clip=true, height=136mm, keepaspectratio]{images/pes_dmf.eps}
 \end{center}
\caption[An overview of the energies of the three mechanistic pathways of photochemical \ce{CO2} reduction in DMF.]{An overview of the energies of the three mechanistic pathways of photochemical \ce{CO2} reduction in DMF. Excimer formation is shown in blue, the planar water-gas shift mechanism in red, the axial water-gas shift in orange, the bicarbonate in purple, and the formate in green.}
\label{fig.apes_dmf}
\end{figure} 

\end{landscape}