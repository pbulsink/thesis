%======================================================================
\chapter{TurboControl}\label{chap.turbocontrol}
\markright{TurboControl}
%======================================================================

The analysis of the mechanism using computational methods required a significant amount of manual set-up and analysis work of computational input and output files. The calculations required 38 different molecules, intermediates, or transition states to be calculated in multiple different environments. These calculations typically require intervention or set-up of intermediate steps to be able to fully elucidate all of the required information, for example, the set-up of frequency calculations after a geometry optimization to ensure ground state or transition state geometries. 

Additionally, while TurboMole contains much faster optimization code for \gls{ac.dft} calculations, the user interface for this software requires a significant learning curve and is based in a user-abrasive interactive text prompt environment. This contrasts to Gaussian 09, which is significantly more popular, partially due to the user-friendliness of the GaussView \gls{ac.gui}, despite concerns about the speed of optimization of large or complex molecules.

These two factors prompted the development of a new program, TurboControl, written in Python, with the goal of combining the user friendliness of GaussView on top of the optimization efficiency of TurboMole. This program allows a user to prepare the input files in GaussView, modify them slightly in any text editor available to them, then run a batch calculation on the files. TurboControl monitors the computational jobs, resubmitting for frequency calculations when optimization is complete if requested, and using the TurboMole tools available when required to ensure ground state or first transition state geometries are discovered. TurboControl then outputs statistics files, providing the energy and first frequency vibration of each molecule in a single output file. Additional commands can instruct TurboControl to run in a solution, do single-point energy calculations, or gather more in-depth thermodynamic information from the outputs using TurboMole's tools. 

TurboControl provides this hands-off management of jobs allowing the user to spend more time in data analysis, experimental work, or to be able to produce data at a significantly higher rate compared to the manual setup and monitoring of jobs. TurboControl also contains a single job initiation script entitled TurboGo to allow for the use of the input files without the overhead of the job managing.

%----------------------------------------------------------------------
\section{Development}
%----------------------------------------------------------------------

TurboControl began as a script for the simple setup and submission of jobs in TurboMole from Gaussian input files. The jobs were submitted to the `wooki' computing cluster available at the University of Ottawa. This computing cluster consists of approximately 1000 CPU cores, running the CentOS distribution of Linux available in the Rocks software distribution. The GridEngine queuing system ensures fair usage of computing cores for each user, and maximizes throughput by managing computational jobs. 

TurboControl initially submitted jobs and monitored their successful completion or failure. Development quickly expanded to include automatic resubmission for frequency analysis of geometry optimization jobs, required to determine the success of finding an energy minima (stationary point) instead of a transition state. As the complexity of the mechanistic study (see \autoref{chap.mech}) increased to include analysis of thermodynamic data, calculation of transition states, calculation in solution phases, and more, the capabilities of TurboControl increased to meed these demands. 

Turbocontrol is now able to handle large batches of input, with jobs of varying complexity and monitor them through successful completion or failure of TurboMole. Analysis of the computational jobs is performed to simply highlight \gls{ac.dft} energies, first computed normal mode frequencies, and thermodynamic properties (if requested). 

%----------------------------------------------------------------------
\section{Usage}
%----------------------------------------------------------------------

TurboControl is available freely on the internet, and is a fully open-source project\autocite{bulsink2014}. The code may be downloaded by anyone and used without prior permission for personal, academic, or commercial purposes, provided that the original software license remains with the code. TurboControl works only with the stated versions of software in the `readme' file available with the code on the internet (and included in \autoref{chap.readme}). The software requires no installation prior to use, and has very few dependencies beyond the typically installed packages on a unix or linux system. 

TurboControl and TurboGo are not quantum mechanical packages themselves, they require a properly licensed installation of TurboMole to perform calculations. Additionally, while GaussView may be used to prepare the input files, modification of the files by hand in a text editor is required to access most features, GaussView is not required for the use of TurboControl. 



