\chapter{New Coordination Geometries for \texorpdfstring{\ce{Re^I}}{Rhenium (I)}}
\markright{New Coordination Geometries for \texorpdfstring{\ce{Re^I}}{Rhenium (I)}} % new right header
%======================================================================
\section{Introduction}
%======================================================================

As mentioned previously in the thesis introduction, \ce{Re^I} compounds have been typically bidentate (\ce{$\kappa$^2}) compounds, even when using a potentially terdentate (\ce{$\kappa$^3}) ligand such as bis(imino)pyridine or terpyridine (refer to \autoref{fig.terdentateligands}). The chemistry of this rhenium $\alpha$-imino complex has been extensively invesigated, with over 1700 references appearing in a structure search for that metal-ligand motif. The extraction of an additional carbonyl and the chelation of the pendant arm of the ligand was attempted to extend the pi system of the ligand and its interaction with the metal centre. This was first demonstrated by prior work in our group for the bis(imino)pyridine ligand\autocite{jurca2013}. 

%======================================================================
\section{Synthesis of Bidentate and Terdentate}
%======================================================================

Similar to the prior work, synthesis began with the production of bidentate \ce{$\kappa$2(terpy)Re(CO)3X} (X = Cl, Br) by coordination of 2,2':6',2''-terpyridine (Sigma) with a \ce{Re(CO)5X} (Strem) starting material in dry toluene at reflux for 4 hours. A bright yellow powder precipitated from solution and was collected by filtration, washed with cold hexanes, and dried \textit{in vacuo} to a good yield\footnote{Experimental details for all compounds can be seen in \nameref{chap:appA}}. These bidentate compounds were used without further purification to produce \ce{$\kappa$3(terpy)Re(CO)2X} (X = Cl, Br) via thermolysis. 

%======================================================================
\section{Anion Exchange}
%======================================================================


%======================================================================
\section{Characterization}
%======================================================================

%----------------------------------------------------------------------
\subsection{NMR Analysis}
%----------------------------------------------------------------------

%----------------------------------------------------------------------
\subsection{X-Ray Crystallography}
%----------------------------------------------------------------------

%----------------------------------------------------------------------
\subsection{Spectroscopy}
%----------------------------------------------------------------------

% - - - - - - - - - - - - - - - - - - - - - - - - - - - - - - - - - - -
\subsubsection{Infrared Spectroscopy}
% - - - - - - - - - - - - - - - - - - - - - - - - - - - - - - - - - - -

% - - - - - - - - - - - - - - - - - - - - - - - - - - - - - - - - - - -
\subsubsection{UV-Vis Spectroscopy}
% - - - - - - - - - - - - - - - - - - - - - - - - - - - - - - - - - - -

%----------------------------------------------------------------------
\subsection{Conclusions}
%----------------------------------------------------------------------

