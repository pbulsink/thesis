%======================================================================
\chapter{Conclusions}
\markright{Conclusions}
%======================================================================

The target \ce{Re^I} terdentate terpyridine compounds were successfully synthesized and fully characterized. The experiments resulted in the first crystalographically verified \textit{mer}-$\kappa^3$-(N, N', N'')-\ce{Re(CO)2X} complexes. Experimental and computational UV-Visible spectra were compared, resulting in a deeper understanding of frontier molecular orbital environments for these complexes. 

The synthesized catalysts were tested for the photoreduction of \ce{CO2}. The $\alpha$-triimino catalysts show no activity for the reduction of \ce{CO2}, in contrast to the known excellent bipyridine compounds, potentially due to the lack of fluorescence of the terdentate complexes, and the chelating ability of the pendant arm in the bidentate 2,2':6,2''-terpyridine ligand. These catalysts may be suitable for electrocatalytic reduction of \ce{CO2}, further investigation will be required.

Reaction mechanisms of the photocatalytic reduction of \ce{CO2} by bipyridine catalysts were studied successfully with \gls{ac.dft} methods, resulting in the proposed new geometry for the production of {CO} with no carbonate or formate anions. This new geometry does not conflict with known experimental studies, yet avoids three-body mechanistic steps previously seen in literature. This geometry provides explanation for previously unexplained phenomenon of \ce{^{13}CO} ligand exchange in few turnovers.

Development of a tool for the rapid submission and automated job monitoring for the TurboMole program facilitated the mechanism investigation by maximizing computational throughput while minimizing set-up and analysis time for the user. The scripts increase the `black box' nature of TurboMole, increasing the usability to include those unfamiliar with the program, and take advantage of the high performance of TurboMole relative to other computational suites. An open source licence and freely available source code mean this project may continue with other developers in the future



