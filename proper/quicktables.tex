\documentclass[10pt,letterpaper]{article}
\usepackage[utf8]{inputenc}
\usepackage{amsmath}
\usepackage{amsfonts}
\usepackage{amssymb}
\usepackage{graphicx}
\usepackage{booktabs}
\usepackage{fixltx2e}
\usepackage{hyperref}
\usepackage{float}
\floatstyle{plaintop}
\restylefloat{table}
\usepackage{pdflscape}
\usepackage{afterpage}

\usepackage[labelfont=bf, labelsep=space, tableposition=top]{caption}
\usepackage{chemscheme}
\usepackage[version=3]{mhchem}
\usepackage{multirow}
\usepackage[para]{threeparttable}


\begin{document}
% Table generated by Excel2LaTeX
\begin{table}[htbp]
  \caption{Selected Distances, Angles, and Torsions for \textbf{2.1}}
  \centering
    \begin{tabular}{ccc}
    \toprule
    \multirow{2}{*}{Bond} & \multicolumn{2}{c}{Distance (\r{A})} \\ \cline{2-3}
     & Experimental & Calculated \\ \midrule
    Re(1)-C(16) & 1.89(1) & 1.916 \\
    Re(1)-C(17) & 1.934(8) & 1.936 \\
    Re(1)-C(18) & 1.90(1) & 1.918 \\
    Re(1)-N(1) & 2.162(6) & 2.197 \\
    Re(1)-N(2) & 2.236(9) & 2.293 \\
    Re(1)-Cl(1) & 2.496(2) & 2.525 \\ 
    C(16)-O(1) & 1.16(1) & \\
    C(17)-O(2) & 1.12(1) & \\
    C(18)-O(3) & 1.15(1) & \\ \midrule
    \multirow{2}{*}{Angle} & \multicolumn{2}{c}{Degrees ($^\circ$)} \\ \cline{2-3}
     & Experimental & Calculated \\ \midrule
    C(16)-Re(1)-C(17) & 87.6(4) & 86.877 \\
    C(16)-Re(1)-C(18) & 88.3(4) & 90.613 \\
    C(17)-Re(1)-C(18) & 87.3(4) & 89.557 \\
    C(16)-Re(1)-N(1) & 96.4(3) & 96.240 \\
    C(17)-Re(1)-N(1) & 174.9(3) & 175.600 \\
    C(18)-Re(1)-N(1) & 95.9(3) & 93.506 \\
    C(16)-Re(1)-N(2) & 169.3(3) & 170.368 \\
    C(17)-Re(1)-N(2) & 101.1(3) & 102.755 \\
    C(18)-Re(1)-N(2) & 98.3(3) & 89.415 \\
    N(2)-Re(1)-N(1) & 74.5(3) & 74.146 \\
    C(16)-Re(1)-Cl(1) & 91.7(3) & 91.453 \\
    C(17)-Re(1)-Cl(1) & 91.7(3) & 94.786 \\
    C(18)-Re(1)-Cl(1) & 179.9(3) & 175.286 \\
    N(1)-Re(1)-Cl(1) & 84.0(2) & 82.058 \\
    N(2)-Re(1)-Cl(1) & 81.6(2) & 87.840 \\
    O(1)-C(16)-Re(1) & 179.6(9) & 178.224 \\
    O(2)-C(17)-Re(1) & 176.0(8) & 176.907 \\ 
    O(3)-C(18)-Re(1) & 177.3(9) & 179.317 \\ \midrule
    \multirow{2}{*}{Torsion} & \multicolumn{2}{c}{Degrees ($^\circ$)} \\ \cline{2-3}
     & Experimental & Calculated \\ \midrule
    N(1)-C(5)-C(6)-N(2) &  16(1) & 15\\
    N(2)-C(10)-C(11)-N(3) & 41(1) & 139\\
    \bottomrule
    \end{tabular}%
  \label{tab.da1}%
\end{table}%


 %1
% Table generated by Excel2LaTeX
\begin{table}[htbp]
  \caption{Selected Distances, Angles, and Torsions for \textbf{2}}
  \centering
    \begin{tabular}{cc}
    \toprule
    \multicolumn{2}{c}{Selected Distances (\r{A})} \\
    \midrule
    Re(1)-C(16) & 1.926(9) \\
    Re(1)-C(17) & 1.975(10) \\
    Re(1)-N(1) & 2.119(7) \\
    Re(1)-N(2) & 2.080(7) \\
    Re(1)-N(3) & 2.126(7) \\
    Re(1)-Cl(1) & 2.489(3) \\
    N(1)-N(3) & 4.14(1) \\ \midrule
    \multicolumn{2}{c}{Selected Angles (deg)} \\ \midrule
    C(16)-Re(1)-C(17) & 91.5(4) \\
    C(16)-Re(1)-N(2) & 173.7(4) \\
    C(17)-Re(1)-N(2) & 94.6(3) \\
    C(16)-Re(1)-N(1) & 103.9(3) \\
    C(17)-Re(1)-N(1) & 92.7(3) \\
    N(2)-Re(1)-N(1) & 77.3(3) \\
    C(16)-Re(1)-N(3) & 101.8(3) \\
    C(17)-Re(1)-N(3) & 91.7(3) \\
    N(2)-Re(1)-N(3) & 76.6(3) \\
    N(1)-Re(1)-N(3) & 153.7(3) \\
    C(16)-Re(1)-Cl(1) & 91.8(3) \\
    C(17)-Re(1)-Cl(1) & 176.5(2) \\
    N(2)-Re(1)-Cl(1) & 82.1(2) \\
    N(1)-Re(1)-Cl(1) & 85.4(2) \\
    N(3)-Re(1)-Cl(1) & 88.7(2) \\
    O(1)-C(16)-Re(1) & 177.9(9) \\
    O(2)-C(17)-Re(1) & 173.2(8) \\ \midrule
    \multicolumn{2}{c}{Selected Torsions (deg)} \\ \midrule
    N(1)-C(5)-C(6)-N(2) & 1(1) \\
    N(2)-C(10)-C(11)-N(3) & -4(1) \\
    \bottomrule
    \end{tabular}%
  \label{tab.da2}%
\end{table}%


 %2
% Table generated by Excel2LaTeX
\begin{table}[htbp]
  \caption{Selected Distances, Angles, and Torsions for \textbf{3}}
  \centering
    \begin{tabular}{ccc}
    \toprule
   \multirow{2}{*}{Bond} & \multicolumn{2}{c}{Distance (\r{A})} \\ \cline{2-3}
     & Experimental & Calculated \\ \midrule
    Re(1)-C(16) & 1.911(3) & \\
    Re(1)-C(17) & 1.890(3) & \\
    Re(1)-C(18) & 1.921(4) & \\
    Re(1)-N(1) & 2.173(3) & \\
    Re(1)-N(2) & 2.232(2) & \\
    Re(1)-Br(1) & 2.6410(4) & \\ \midrule
    \multirow{2}{*}{Angle} & \multicolumn{2}{c}{Degrees ($^\circ$)} \\ \cline{2-3}
     & Experimental & Calculated \\ \midrule
    C(16)-Re(1)-C(17) & 89.1(1) & \\
    C(16)-Re(1)-C(18) & 85.9(1) & \\
    C(16)-Re(1)-N(1) & 97.9(1) & \\
    C(17)-Re(1)-N(1) & 92.5(1) & \\
    C(18)-Re(1)-N(1) & 175.4(1) & \\
    C(16)-Re(1)-N(2) & 171.2(1) & \\
    C(17)-Re(1)-N(2) & 96.0(1) & \\
    C(18)-Re(1)-N(2) & 101.3(1) & \\
    N(1)-Re(1)-N(2) & 74.7(1) & \\
    C(16)-Re(1)-Br(1) & 92.7(1) & \\
    C(17)-Re(1)-Br(1) & 177.6(1) & \\
    C(18)-Re(1)-Br(1) & 91.6(1) & \\
    N(1)-Re(1)-Br(1) & 85.74(7) & \\
    N(2)-Re(1)-Br(1) & 82.07(7) & \\
    O(1)-C(16)-Re(1) & 178.6(3) & \\
    O(2)-C(17)-Re(1) & 179.5(3) & \\
    O(3)-C(18)-Re(1) & 179.9(3) & \\ \midrule
    \multicolumn{2}{c}{Selected Torsions (deg)} \\ \midrule
    N(1)-C(6)-C(1)-N(2) & -15.4(4) & \\
    N(2)-C(5)-C(11)-N(3) & 141.1(3) & \\
    \bottomrule
    \end{tabular}%
  \label{tab.da3}%
\end{table}%


 %3
% Table generated by Excel2LaTeX
\begin{table}[htbp]
  \caption{Selected Distances, Angles, and Torsions for \textbf{5}}
  \centering
    \begin{tabular}{cccccc}
    \toprule
    \multicolumn{3}{c}{Axial CN} & \multicolumn{3}{c}{Planar CN} \\ \midrule
    \multirow{2}{*}{Bond} & \multicolumn{2}{c}{Distance (\r{A})} & \multirow{2}{*}{Bond} & \multicolumn{2}{c}{Distance (\r{A})} \\ \cline{2-3} \cline {5-6}
     & Exp. & Calc. & & Exp. & Calc. \\ \midrule
    Re(2)-C(35) & 2.148(7) & 2.13963 & Re(1)-C(19) & 2.105(8) & 1.98769 \\
    Re(2)-C(36) & 1.926(6) & 1.94011 & Re(1)-C(16) & 1.928(5) & 2.09197 \\
    Re(2)-C(37) & 1.954(7) & 1.96758 & Re(1)-C(18) & 1.96(1) & 2.00792 \\
    Re(2)-C(38) & 1.902(9) & 1.91853 & Re(1)-C(17) & 1.918(7) & 1.90499 \\
    Re(2)-N(5) & 2.242(7) & 2.28998 & Re(1)-N(1) & 2.253(5) & 2.32197 \\
    Re(2)-N(6) & 2.168(5) & 2.20279 & Re(1)-N(2) & 2.176(4) & 2.18806 \\
    C(35)-N(8) & 1.138(9) & 1.16104 & C(19)-O(3) & 1.17(1) & 1.14703 \\
    C(36)-O(4) & 1.145(8) & 1.15044 & C(16)-N(4) & 1.149(7) & 1.16100 \\
    C(37)-O(5) & 1.151(9) & 1.15134 & C(18)-O(2) & 1.14(1) & 1.14276 \\
    C(38)-O(6) & 1.17(1) & 1.15368 & C(17)-O(1) & 1.130(8) & 1.15781 \\ \midrule
    \multirow{2}{*}{Angle} & \multicolumn{2}{c}{Degrees ($^\circ$)} & \multirow{2}{*}{Angle} & \multicolumn{2}{c}{Degrees ($^\circ$)} \\ \cline{2-3} \cline {5-6}
     & Exp. & Calc. & & Exp. & Calc. \\ \midrule
    C(36)-Re(2)-C(38) & 87.7(3) & 87.273 & C(16)-Re(1)-C(17) & 87.8(3) & 90.158 \\
    C(36)-Re(2)-C(37) & 88.0(3) & 89.890 & C(16)-Re(1)-C(18) & 87.0(3) & 84.822 \\
    C(36)-Re(2)-C(35) & 92.1(3) & 93.356 & C(16)-Re(1)-C(19) & 92.5(3) & 88.356 \\
    C(38)-Re(2)-C(37) & 88.5(3) & 90.973 & C(17)-Re(1)-C(18) & 88.7(3) & 88.453 \\
    C(38)-Re(2)-C(35) & 90.8(3) & 91.628 & C(17)-Re(1)-C(19) & 90.5(3) & 87.745 \\
    C(37)-Re(2)-C(35) & 179.2(3) & 175.933 & C(18)-Re(1)-C(19) & 179.1(3) & 172.179 \\
    C(36)-Re(2)-N(5) & 100.6(3) & 102.576 & C(16)-Re(1)-N(1) & 102.2(2) & 98.105 \\
    C(36)-Re(2)-N(6) & 174.2(3) & 175.708 & C(16)-Re(1)-N(2) & 175.9(2) & 172.047 \\
    C(38)-Re(2)-N(5) & 169.3(3) & 170.146 & C(17)-Re(1)-N(1) & 168.3(3) & 170.509 \\
    C(38)-Re(2)-N(6) & 96.6(3) & 96.171 & C(17)-Re(1)-N(2) & 95.9(3) & 97.544 \\
    C(37)-Re(2)-N(5) & 98.4(2) & 89.360 & C(18)-Re(1)-N(1) & 97.7(3) & 88.487 \\
    C(37)-Re(2)-N(6) & 96.0(2) & 92.605 & C(18)-Re(1)-N(2) & 94.8(3) & 93.374 \\
    C(35)-Re(2)-N(5) & 82.3(2) & 87.543 & C(19)-Re(1)-N(1) & 83.2(2) & 96.317 \\
    C(35)-Re(2)-N(6) & 83.9(2) & 84.008 & C(19)-Re(1)-N(2) & 85.7(2) & 93.899 \\
    N(5)-Re(2)-N(6) & 74.7(2) & 73.977 & N(1)-Re(1)-N(2) & 73.9(2) & 73.675 \\
    O(6)-C(38)-Re(2) & 179.4(7) & 178.027 & O(1)-C(17)-Re(1) & 178.2(7) & 177.623 \\
    O(5)-C(37)-Re(2) & 175.5(6) & 179.414 & O(2)-C(18)-Re(1) & 172.0(7) & 176.452 \\ 
    N(8)-C(35)-Re(2) & 178.0(6) & 176.457 & O(3)-C(19)-Re(1) & 178.0(6) & 176.552 \\
    O(4)-C(36)-Re(2) & 179.0(7) & 177.313 & N(4)-C(16)-Re(1) & 178.7(6) & 178.113 \\ \midrule
    \multirow{2}{*}{Torsion} & \multicolumn{2}{c}{Degrees ($^\circ$)} & \multirow{2}{*}{Torsion} & \multicolumn{2}{c}{Degrees ($^\circ$)} \\ \cline{2-3} \cline {5-6}
     & Exp. & Calc. & & Exp. & Calc. \\ \midrule
    N(5)-C(20)-C(25)-N(6) & 14.5(9) & 13.735 & N(1)-C(1)-C(6)-N(2) & 12.5(8) & 14.777 \\
    N(5)-C(24)-C(30)-N(7) & 41(1) & 135.774 & N(1)-C(5)-C(11)-N(3) & 43.7(9) & 137.014 \\
    \bottomrule
    \end{tabular}%
  \label{tab.da5}%
\end{table}%


 %4
% Table generated by Excel2LaTeX from sheet 'Sheet3'
\begin{table}[htbp]
  \centering
  \caption{Selected Distances and Angles for \textbf{8\ce{.CH3CN}}}
    \begin{tabular}{cc}
    \toprule
    \multicolumn{2}{c}{Selected Distances (\r{A})} \\
    \midrule
    Re(1)-C(16) & 1.889(4) \\
    Re(1)-C(17) & 1.885(3) \\
    Re(1)-N(1) & 2.091(3) \\
    Re(1)-N(2) & 2.135(3) \\
    Re(1)-N(3) & 2.131(3) \\
    Re(1)-N(4) & 2.160(3) \\ \midrule
    \multicolumn{2}{c}{Selected Angles (deg)} \\ \midrule
    C(16)-Re(1)-C(17) & 87.69(16) \\
    C(16)-Re(1)-N(1) & 175.95(12) \\
    C(17)-Re(1)-N(1) & 96.35(12) \\
    C(16)-Re(1)-N(3) & 103.81(13) \\
    C(17)-Re(1)-N(3) & 94.03(12) \\
    N(1)-Re(1)-N(3) & 76.20(10) \\
    C(16)-Re(1)-N(2) & 103.58(13) \\
    C(17)-Re(1)-N(2) & 93.73(12) \\
    N(1)-Re(1)-N(2) & 75.99(10) \\
    N(3)-Re(1)-N(2) & 151.77(11) \\
    C(16)-Re(1)-N(4) & 90.50(14) \\
    C(17)-Re(1)-N(4) & 178.10(12) \\
    N(1)-Re(1)-N(4) & 85.46(10) \\
    N(3)-Re(1)-N(4) & 86.94(10) \\
    N(2)-Re(1)-N(4) & 86.15(10) \\
    O(1)-C(16)-Re(1) & 179.1(3) \\
    O(2)-C(17)-Re(1) & 178.0(3) \\
    \bottomrule
    \end{tabular}%
  \label{tab.da8}%
\end{table}%

 %5
% Table generated by Excel2LaTeX
\begin{table}[htbp]
  \caption[Selected Distances, Angles and Torsions from a Mn Analogue]{Selected Distances, Angles, and Torsions for \texorpdfstring{\ce{$\kappa^2$(terpy)Mn(CO)3Br}}{Bidentate Terpyridine Tricarbonyl Mn Br} from Compain et. al.}
  \centering
    \begin{tabular}{cc}
    \toprule
    \multicolumn{2}{c}{Selected Distances (\r{A})} \\
    \midrule
	Mn(1)-N(1) & 2.045(1) \\
	Mn(1)-N(2) & 2.105(2) \\
	N(1)-N(2) & 2.636(2) \\ \midrule
    \multicolumn{2}{c}{Selected Angles (deg)} \\ \midrule
    N(1)-Mn(1)-N(2) & 78.84(6) \\ \midrule
    \multicolumn{2}{c}{Selected Torsions (deg)} \\ \midrule
    N(1)-C(8)-C(9)-N(2) & -16.5(2) \\
    N(2)-C(13)-C(14)-N(3) & 143.2(2) \\
    \bottomrule
    \end{tabular}%
  \label{tab.damnbr}%
\end{table}%


 %6
% Table generated by Excel2LaTeX
\afterpage{
\begin{landscape}
\begin{table}[p]
  \caption{Crystal data and structure refinement for compounds \textbf{1}, \textbf{3}, \textbf{5}, and \textbf{7}}
  \centering
  \begin{tabular}{rp{3.2cm}p{3.2cm}p{3.2cm}p{3.2cm}}
    \toprule
    Compound & \textbf{1} & \textbf{3} & \textbf{5} & \textbf{7} \\
    \cmidrule(l){2-5} 
    Empirical formula & \ce{C19H11N3O3ReCl} & \ce{C19H11N3O3ReBr} & \ce{C20H11N4O3Re} & \ce{C22H14N4O6F3SRe} \\
    Formula weight (g/mol) & 538.96 & 583.41 & 530.04 & 693.63 \\
    Temperature (K) & 200(2) & 200 & 200 & 200 \\
    Wavelength (\r{A})  & 0.71073 & 0.71073 & 0.71073 & 0.71073 \\
    Crystal System & Triclinic & Monoclinic & Triclinic & \\
    Space Group & P-1 & C2/c & P-1 &  \\
    a (\r{A}) & 9.8736(4) & 31.1537(7) & 9.9196(9) &  \\
    b (\r{A}) & 14.8202(4) & 7.1176(2) & 14.9902(14) &  \\
    c (\r{A}) & 16.3472(4) & 16.8519(4) & 16.5187(15) &  \\
    $\alpha$ (deg) & 69.2890(10) & 90.000 & 68.363(2) &  \\
    $\beta$ (deg) & 80.801(2) & 111.0230(10) & 80.929(2) &  \\
    $\gamma$ (deg) & 79.836(2) & 90.000 & 79.975(2) &  \\
    Volume (\r{A}\textsuperscript{3}) & 2190.00(12) & 3488.00 & 2236.6(4) & \\
    Z, r (calc) (Mg/m\textsuperscript{3}) & 2, 1.997 & 8, 2.222 & 2, 1.927 & \\
    Absorption coefficient (mm\textsuperscript{-1}) & 6.063 & 9.282 & 5.821 & \\
    Absorption correction  & \multicolumn{4}{c}{Semi-empirical from equivalents} \\
    Final R indices [I$\geq$2$\sigma$(I)] & R1 = 0.0397,\newline wR2 = 0.0839 & R1 = 0.0232,\newline wR2 = 0.0614 & R1 = 0.0390,\newline wR2 = 0.0921 & \\
    R indices (all data) & R1 = 0.0604,\newline wR2 = 0.0951 & R1 = 0.0285,\newline wR2 = 0.0642 & R1 = 0.0500,\newline wR2 = 0.0961 & \\
    \bottomrule
  \end{tabular} 
  \label{tab.bidxraycp}%
\end{table}%
\end{landscape}
\clearpage
} %7
% Table generated by Excel2LaTeX
\begin{table}[htb]
\begin{center}
  \caption{Crystal data and structure refinement for compounds \textbf{2.2} and \textbf{2.8}}
    \begin{tabular}{rp{3.2cm}p{3.2cm}}
    \toprule
    Compound & \textbf{2.2} & \textbf{2.8} \\
    \cmidrule(l){2-3} 
    Empirical formula& \ce{C18H11N3O2ReCl} & \ce{C21H14N4O5F3SRe} \\
    Formula weight (g/mol) & 510.95 & 665.61 \\
    Temperature (K) & 200 & 200 \\
    Wavelength (\r{A}) & 0.71073 & 0.71073 \\
    Crystal System & Triclinic & Triclinic \\
    Space Group & P-1 & P-1 \\
    a (\r{A}) & 8.5275(3) & 8.5745(4) \\
    b (\r{A}) & 14.2421(5) & 11.9805(5) \\
    c (\r{A}) & 17.4637(6) & 13.0970(5) \\
    $\alpha$ (deg) & 77.948(2) & 79.748(2) \\
    $\beta$ (deg) & 85.684(2) & 81.106(2) \\
    $\gamma$ (deg) & 79.890 & 88.091(2) \\
    Volume (\r{A}\textsuperscript{3}) & 2041.79(12) & 1307.99(10) \\
    Z, r (calc) (Mg/m\textsuperscript{3}) & 4, 2.050 & 2, 1.993 \\
    Absorption coefficient (mm\textsuperscript{-1}) & 6.494 & 5.094 \\
    Absorption correction  & \multicolumn{2}{c}{Semi-empirical from equivalents} \\
    Final R indices [I$\geq$2$\sigma$(I)] & R1 = 0.0636,\newline wR2 = 0.1018 & R1 = 0.0294,\newline wR2 = 0.0673 \\
    R indices (all data) & R1 = 0.0985,\newline wR2 = 0.1110 & R1 = 0.0366,\newline wR2 = 0.0700 \\
    \bottomrule
    \end{tabular}%
\end{center}
  \label{tab.terxraycp}%
\end{table}% %8
% Table generated by Excel2LaTeX
\begin{table}[!h]
\centering
 \begin{threeparttable}
  \caption{Solvated and gas phase energy differences between Axial \& Trans geometries of \ce{$\kappa$^x-(terpy)-Re(CO)_{5-x}CN} (x=2,3)}
    \begin{tabular}{cccc}
    \toprule
    \multicolumn{2}{c}{\underline{Bidentate}} & \multicolumn{2}{c}{\underline{Terdentate}} \\
    E(gas)\tnote{a} & E(solution)\tnote{b} & E(gas)\tnote{a} & E(solution)\tnote{b} \\ \midrule
    14.70 & 11.87 & 16.28 & 16.43 \\
    \bottomrule
    \end{tabular}%
    \begin{tablenotes}
    \item [a] B3LYP SCF energy in kcal/mol.
    \item [b] B3LYP SCF energy in kcal/mol, with PCM solvation in acetonitrile.
    \end{tablenotes}
  \label{tab.cneng}%
 \end{threeparttable}
\end{table}%


 %9
% Table generated by Excel2LaTeX
\begin{table}[!htb]
\centering
 \begin{threeparttable}
  \caption[Gas phase and solvated energies for the `carbonate' mechanism]{Gas phase and solvated energies of compounds, transition states and intermediates in the `carbonate' mechanism}
    \begin{tabular}{llrrr}
    \toprule
    Molecule & Label & E (gas)\tnote{a} & E (solution)\tnote{b} & E (solvation)\tnote{c} \\
    \midrule
    Catalyst - \ce{CO2} & \textbf{4.11}  & -1103.009 & -1103.016 & 4.47 \\
    \ce{CO2} Linked Dimer & \textbf{4.12} & -2017.443 & -2017.481 & 24.35 \\
    \ce{CO2} Addition & \textbf{TS4.13} & -2206.102 & -2206.146 & 27.78 \\
    \ce{C2O4} Linked Dimer & \textbf{4.14} & -2206.126 & -2206.172 & 28.98 \\
    5 Member Ringed Dimer & \textbf{TS4.15} & -2206.014 & -2206.062 & 29.87 \\
    \ce{CO3} Linked Dimer & \textbf{4.16} & -2092.678 & -2092.726 & 29.75 \\
    Bicarbonate Catalyst Cation & \textbf{4.17} & -1178.065 & -1178.120 & 34.24 \\
    Bicarbonate Anion & \textbf{4.18} & -264.485 & -264.497 & 7.20 \\
    Open Site Cation & \textbf{4.19} & -914.106 & -914.187 & 50.75 \\
    \bottomrule
    \end{tabular}%
    \begin{tablenotes}
    \item [a] TPSS energy in hartrees.
    \item [b] TPSS energy in hartrees with COSMO solvation in DMF.
    \item [c] TPSS solvation energy in kcal/mol (E(gas) - E(solution)).
    \end{tablenotes}
  \label{tab.carbenergy}%
 \end{threeparttable}
\end{table}%


 %10
% Table generated by Excel2LaTeX
\begin{table}[!htb]
\centering
 \begin{threeparttable}
  \caption{Energies for the reaction steps in the `formate' pathway}
    % Table generated by Excel2LaTeX from sheet 'Tex Charts
    \begin{tabular}{rrrr}
    \toprule
    Description & Steps & Energy(gas)\tnote{a} & Energy(dmf)\tnote{b} \\
    \midrule
    Formation of Radical Anion & 4.1   & -39.271301 & -67.827267 \\
    Open site catalyst plus cl- & 4.2   & 50.8169125 & 15.4408871 \\
    Reconfiguration of TEA & 4.3   & -164.60991 & -118.34007 \\
    \midrule
    Addition of CO2 to open site & 4.4   & -0.2501423 & 6.37310903 \\
    addition of second cat to CO2 & 4.5   & 118379.026 & 118375.044 \\
    Insertion of CO2 & 4.6   & \#VALUE! & -8.5491267 \\
    relaxation of co2 insertion & 4.7   & \#VALUE! & -0.6637971 \\
    rearrange to 4ring dimer & 4.8   & \#VALUE! & \#VALUE! \\
    relax to long & 4.9   & \#VALUE! & \#VALUE! \\
    rearrangement to 5ring dimer & 4.1   & \#VALUE! & 22.4628711 \\
    relax to final & 4.11  & \#VALUE! & -24.839557 \\
    break apart & 4.12  & 317.951051 & 262.714812 \\
    return to ground states & 4.13  & -613.91666 & -426.82371 \\
    \bottomrule
    \end{tabular}%
    \begin{tablenotes}
    \item [a] TPSS SCF energy in kcal/mol.
    \item [b] TPSS SCF energy in kcal/mol with COSMO solvation in DMF.
    \end{tablenotes}
  \label{tab.carbrxn}%
 \end{threeparttable}
\end{table}%
 %11
% Table generated by Excel2LaTeX
\begin{table}[!htb]
\centering
 \begin{threeparttable}
  \caption[Gas phase and solvated energies for the `formate' mechanism]{Gas phase and solvated energies of compounds, transition states and intermediates in the `formate' mechanism}
    \begin{tabular}{llrrr}
    \toprule
    Molecule & Label & E (gas)\tnote{a} & E (solution)\tnote{b} & E (solvation)\tnote{c} \\
    \midrule
    Proton Transfer TS & 4.21  & -1206.302997 & -1206.32707 & 15.10 \\
    Catalyst Hydride & 4.22  & -914.9204746 & -914.9448354 & 15.29 \\
    \ce{CO2} Insertion TS & 4.23  & -1103.581201 & -1103.619960 & 24.32 \\
    Catalyst Formate & 4.24  & -1103.635283 & -1103.665628 & 19.04 \\
    Formate Anion & 4.25 & -189.3051464 & -189.4151284 & 69.01 \\
    Open Site Cation & 4.27  & -914.1064097 & -914.1872844 & 50.75 \\
    \bottomrule
    \end{tabular}%
    \begin{tablenotes}
    \item [a] TPSS SCF energy in hartrees.
    \item [b] TPSS SCF energy in hartrees with COSMO solvation in DMF.
    \item [c] TPSS solvation energy in kcal/mol (E(gas) - E(solution)).
    \end{tablenotes}
  \label{tab.formenergy}%
 \end{threeparttable}
\end{table}%


 %12
% Table generated by Excel2LaTeX
\begin{table}[!htb]
\centering
 \begin{threeparttable}
  \caption{Energies for the reaction steps in the `formate' pathway}
    % Table generated by Excel2LaTeX from sheet 'Tex Charts
    \begin{tabular}{r@{ $\rightarrow$ }lrr}
    \toprule
    \multicolumn{2}{c}{Steps} & Energy(gas)\tnote{a} & Energy(dmf)\tnote{b} \\
    \midrule
    \textbf{4.03} + \textbf{4.07} & \textbf{TS4.21} & -44.98 & -48.46 \\
    \textbf{TS4.21} & \textbf{4.22} + \textbf{4.26} & 22.43 & 18.60 \\
    \textbf{4.22} + \textbf{4.09} & \textbf{TS4.23} & 21.24 & 14.02 \\
    \textbf{TS4.23} & \textbf{4.24} & -33.94 & -28.66 \\
    \textbf{4.24} & \textbf{4.25} + \textbf{TS4.27} & 140.39 & 39.67 \\
    \textbf{TS4.27} + \textbf{4.04} & \textbf{4.01} & -141.77 & -36.37 \\
    \bottomrule
    \end{tabular}%
    \begin{tablenotes}
    \item [a] TPSS SCF energy in kcal/mol.
    \item [b] TPSS SCF energy in kcal/mol with COSMO solvation in DMF.
    \end{tablenotes}
  \label{tab.formrxn}%
 \end{threeparttable}
\end{table}%

 %13
% Table generated by Excel2LaTeX
\begin{table}[!htb]
\centering
 \begin{threeparttable}
  \caption{Energies for the reaction steps in the `equatorial' geometry}
    % Table generated by Excel2LaTeX from sheet 'Tex Charts
    \begin{tabular}{rrrr}
    \toprule
    Description & Steps & Energy(gas)\tnote{a} & Energy(dmf)\tnote{b} \\
    \midrule
    & 4.01 \ce{->} 4.01\textsuperscript{3MLCT} & 42.81191677 &	53.40083976 \\
    & 4.01\textsuperscript{3MLCT} + 4.05 \ce{->} 4.02 + 4.06 & 81.44966532	& -5.80393275 \\
    & 4.02 \ce{->} 4.03 + 4.04 & 50.816912 & 15.440887 \\
    & 4.06 + 4.05 \ce{->} 4.07 + 4.08 & -1.077024 & -2.915901 \\
    \midrule
    Migration of Open Site & 3 \ce{->} 38 (3.4) & 23.36742627 & 19.7662463 \\
    Addition of \ce{CO2} to open site & 38, 35 \ce{->} 14 (3.4) & 4.789871838 & 1.562073338 \\
    H transfer to \ce{CO2} & 14, 28 \ce{->} 15 (3.5) & -36.36053213 & -49.1348917 \\
    \ce{CO2H} equatorial relaxation & 15 \ce{->} 16, 30 (3.6) & -3.388580017 & 8.343976667 \\
    \ce{COOH2} ts & 16, 29 \ce{->} 17, 26 (3.7) & -18.2531325 & 0.641599428 \\
    \ce{CO4} + and \ce{H2O} & 17 \ce{->} 13, 36 (3.8) & 3.732973932 & -0.992257078 \\
    dissassotiation of \ce{CO} & 13 \ce{->} 34, 8 (3.9) & 40.89377695 & 35.56739959 \\
    Reformation of Catalyst & 8, 31 \ce{->} 1 (3.10) & -141.7699459 & -36.36774459 \\
    \bottomrule
    \end{tabular}%
    \begin{tablenotes}
    \item [a] TPSS SCF energy in kcal/mol.
    \item [b] TPSS SCF energy in kcal/mol with COSMO solvation in DMF.
    \end{tablenotes}
  \label{tab.siderxn}%
 \end{threeparttable}
\end{table}%


 %14
% Table generated by Excel2LaTeX
\begin{table}[!htb]
\centering
 \begin{threeparttable}
  \caption[Gas phase and solvated energies for mechanism reactants and products]{Gas phase and solvated energies of mechanism reactants and products.}
    \begin{tabular}{llrrr}
    \toprule
    Molecule & Label & E (gas)\tnote{a} & E (solution)\tnote{b} & E (solvation)\tnote{c} \\
    \midrule
    Ground State & 4.01 & -1374.621419 & -1374.651099 & 18.62 \\
    Radical Anion & 4.02 & -1374.684002 & -1374.759190 & 47.18 \\
    Open Site Excimer & 4.03 & -914.3139376 & -914.3287245 & 9.28 \\
    Chlorine Anion & 4.04  & -460.2890817 & -460.4058583 & 73.28 \\
    Triethylamine (TEA) & 4.05  & -292.3051496 & -292.3854033 & 50.36 \\
    Radical Cation TEA & 4.06  & -292.3051496 & -292.3854033 & 50.36 \\
    Deprotonated TEA Radical & 4.07  & -291.9173706 & -291.9211226 & 2.35 \\
    Triethylammonia & 4.08  & -292.9552538 & -293.0382729 & 52.09 \\
    Carbon Dioxide & 4.09  & -188.6945676 & -188.6974631 & 1.82 \\
    Carbon Monoxide & 4.10  & -113.3744946 & -113.3754466 & 0.60 \\
    Diethylaminoethene & 4.26  & -291.3467768 & -291.3525868 & 3.64 \\
    \bottomrule
    \end{tabular}%
    \begin{tablenotes}
    \item [a] TPSS SCF energy in hartrees.
    \item [b] TPSS SCF energy in hartrees with COSMO solvation in DMF.
    \item [c] TPSS solvation energy in kcal/mol (E(gas) - E(solution)).
    \end{tablenotes}
  \label{tab.supenergy}%
 \end{threeparttable}
\end{table}%


 %15
% Table generated by Excel2LaTeX
\begin{table}[!htb]
\centering
 \begin{threeparttable}
  \caption{Energies for the reaction steps in the photoinduced excimer formation pathway}
    \begin{tabular}{r@{ $\rightarrow$ }lrr}
    \toprule
    \multicolumn{2}{c}{Steps} & Energy(gas)\tnote{a} & Energy(dmf)\tnote{b} \\
    \midrule
    \textbf{4.01} & \textbf{\textsuperscript{3}4.01\textsuperscript{MLCT}} & 42.81 & 53.40 \\
    \textbf{4.01\textsuperscript{3MLCT}} + \textbf{4.05} & \textbf{4.02} + \textbf{4.06} & 81.45 & -5.80 \\
    \textbf{4.02} & \textbf{4.03} + \textbf{4.04} & 50.82 & 15.44 \\
    \textbf{4.06} + \textbf{4.05} & \textbf{4.07} + \textbf{4.08} & -1.08 & -2.92 \\
    \bottomrule
    \end{tabular}%
    \begin{tablenotes}
    \item [a] TPSS energy in kcal/mol.
    \item [b] TPSS energy in kcal/mol with COSMO solvation in DMF.
    \end{tablenotes}
  \label{tab.suprxn}%
 \end{threeparttable}
\end{table}%


 %16
% Table generated by Excel2LaTeX
\begin{table}[!htb]
\centering
 \begin{threeparttable}
  \caption[Gas phase and solvated energies for the `water-gas shift' mechanism]{Gas phase and solvated energies of compounds, transition states and intermediates in the `water-gas shift' mechanism}
    \begin{tabular}{rrrr}
    \toprule
    Label & E (gas)\tnote{a} & E (solution)\tnote{b} & E (solvation)\tnote{c} \\
    \midrule
    4.01  & -1374.621419 & -1374.651099 & 18.62 \\
    4.03  & -914.3139376 & -914.3287245 & 9.28 \\
    4.04  & -460.2890817 & -460.4058583 & 73.28 \\
    4.07  & -291.9173706 & -291.9211226 & 2.35 \\
    4.10  & -113.3744946 & -113.3754466 & 0.60 \\
    4.26  & -291.3467768 & -291.3525868 & 3.64 \\
    4.31  & -1103.008904 & -1103.016031 & 4.47 \\
    4.32  & -1103.610331 & -1103.640451 & 18.90 \\
    4.33  & -1104.018352 & -1104.093194 & 46.96 \\
    4.34  & -1102.963633 & -1102.992198 & 17.92 \\
    4.35  & -1103.597572 & -1103.625739 & 17.68 \\
    4.36  & -1104.016748 & -1104.092840 & 47.75 \\
    4.37  & -76.46413339 & -76.47581393 & 7.33 \\
    4.38  & -1027.546073 & -1027.619412 & 46.02 \\
    4.39  & -1394.956766 & -1394.991144 & 21.57 \\
    4.40  & -1394.979805 & -1394.970048 & -6.12 \\
    4.41  & -914.2766988 & -914.2972247 & 12.88 \\
    \bottomrule
    \end{tabular}%
    \begin{tablenotes}
    \item [a] TPSS SCF energy in hartrees.
    \item [b] TPSS SCF energy in hartrees with COSMO solvation in DMF.
    \item [c] TPSS solvation energy in kcal/mol (E(gas) - E(solution)).
    \end{tablenotes}
  \label{tab.wgsenergy}%
 \end{threeparttable}
\end{table}%


 %17
% Table generated by Excel2LaTeX
\begin{table}[!htb]
\centering
 \begin{threeparttable}
  \caption{Energies for the reaction steps in the `water-gas shift' mechanism}
    \begin{tabular}{rrrr}
    \toprule
    Description & Steps & Energy(gas)\tnote{a} & Energy(dmf)\tnote{b} \\
    \midrule
    & 4.01 \ce{->} 4.01\textsuperscript{3MLCT} & 42.81191677 &	53.40083976 \\
    & 4.01\textsuperscript{3MLCT} + 4.05 \ce{->} 4.02 + 4.06 & 81.44966532	& -5.80393275 \\
    & 4.02 \ce{->} 4.03 + 4.04 & 50.816912 & 15.440887 \\
    & 4.06 + 4.05 \ce{->} 4.07 + 4.08 & -1.077024 & -2.915901 \\
    \midrule
    Addition of \ce{CO2} to open site & 3, 35 \ce{->} 9 (2.5) & -0.250142337 & 6.373109033 \\
    H transfer to \ce{CO2} TS & 9, 28 \ce{->} 10 (2.6) & -34.68146588 & -46.71680199 \\
    \ce{CO2H} axial relaxation & 10 \ce{->} 11, 30 (2.7) & 15.33334974 & 11.64963925 \\
    \ce{COOH2} ts & 11, 29 \ce{->} 12, 26 (2.8) & -11.62501284 & 10.08113151 \\
    \ce{CO4} + and \ce{H2O} & 12 \ce{->} 13, 36 (2.9) & 5.111298704 & -1.20033086 \\
    dissassotiation of \ce{CO} & 13 \ce{->} 8, 34 (2.10) & 40.89377695 & 35.56739959 \\
    Reformation of Catalyst & 8, 31 \ce{->} 1 (2.11) & -141.7699459 & -36.36774459 \\
    \bottomrule
    \end{tabular}%
    \begin{tablenotes}
    \item [a] TPSS SCF energy in kcal/mol.
    \item [b] TPSS SCF energy in kcal/mol with COSMO solvation in DMF.
    \end{tablenotes}
  \label{tab.wgsrxn}%
 \end{threeparttable}
\end{table}%


 %18
\end{document}